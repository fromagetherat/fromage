\subsubsection{Uma --- Fringe-Toed Lizards}
\begin{center}
\begin{longtabu} to \textwidth {| | p{3.5cm} | X | |}

	\hline
	Taxonomy/Ancestry &
	
	 \\
	\hline
	Size & 
	
	\\
	\hline
	Color &
	brown/tan/grayish coloration w/ contrasting pattern of black splotches/eyespots on dorsal. pale underside w/ black bars on underside of tail, black mark on lower sides.
	 \\
	\hline
	Anatomy &
	\begin{itemize}[noitemsep]
		\item fringe on toes helps them run quickly over sand, stop from sinking
		\item dorsal surface = velvety, intricate markings w/ granular scales to help bury in sand
		\item upper jaws overlap w/ lower and nostrils can be closed at will to prevent intrusion of sand particles
			\begin{itemize}[noitemsep]
				\item flaps close against ear openings
				\item lower eyelids have interlocking scales 
			\end{itemize}
	\end{itemize}

	\centeredgraphics{lacertila/phrynosomatidae/uma/feet}{0.15}
	 \\
	\hline
	Dimorphism & 
	males have 2 enlarged postanal scales, distinct femoral pores, hemipenal bulge. females have more pronounced pinkish coloration on sides during mating season
	\\
	\hline
	Behavior & 
	\begin{itemize}[noitemsep]
		\item run quickly across dunes
		\item parietal eye thought to help self-monitor level of solar radiation
		\item often basks w/ only head above sand until body temp warms enough to unbury body and start activity
		\item typically burrow into sand or under bush in defense
		\item not communal, but a hierarchy will develop among individuals forced to coexist within a confined space; dominance is established through aggression displays and fighting, and results in a hierarchy in which a single dominant male has several equal subordinates, with additional tiers of less dominant individuals below the subordinates
		\item both male and female Uma inornata maintain home territories; the territory of an average adult male is about 1070 square meters, while that of an average adult female is about 437 square meters. Territory size has been found to be proportional to snout vent length, with younger lizards maintaining proportionally smaller territories than their adult counterparts
		\item frightened lizard may either retreat into a nearby rodent burrow or dive beneath the sand. In addition, lizards in the genus Uma have been observed running bipedally over the sand when fleeing predators at high speeds.
	\end{itemize}
	\\
	\hline
	Habitat & 
	low desert areas w/ fine, loose sand, including dunes, flats with sandy hummocks formed around the bases of vegetation, washes, and the banks of rivers. needs fine, loose sand for burrowing.
	\\
	\hline
	Distribution & 
	range thru SE California, SW Arizona, extend into NW Sonora and NE Baja California.
	\\
	\hline
	Feeding Ecology & 
	primarily insectivores: ants, beetles, grasshoppers, caterpillars. also eat flower buds, stems, leaves, plant seeds. varies on an annual cycle, with a primary diet of flowers and plant-dwelling arthropods during the spring and a primary diet of ground-dwelling arthropods and leaves during the summer. During the month of May (the peak of the breeding season) male and female diets differ significantly, with females specializing in energy maximizing foods (anything with high nutritional value) and males specializing in time minimizing foods (usually easily located flowers and plant matter). individuals will also consume their own or others' shed skins if encountered.
	\\
	\hline
	Reproductive Biology & 
	lays 1-5 eggs from May-July. Male reproductive success has been found to be influenced heavily by precipitation and food supply; in years of low winter precipitation and inadequate nutrition, testes of Uma inornata do not become reproductively active. sexually mature at 2 years of age. no parental care.
	\\
	\hline
	Ecological Role &
	eaten by the American Kestrel and Loggerhead Shrike.
	\\
	\hline
	Conservation Status & 
	The Coachella Valley fringe-toed lizard is listed on the U.S. Endangered Species Act List as threatened, and on the IUCN Red List as endangered (Uma inornata is not listed on the CITES appendices). Causes of the fringe-toed lizard's status are numerous, and include habitat fragmentation, drought, and scouring. Both the 3,709 acre Coachella Valley National Wildlife Refuge and the adjacent 16,405 acre Coachella Valley Preserve have been established to protect the remaining areas in which Uma inornata still occurs. In addition, concerns about movement of sand off from protected habitat areas are being addressed through the Coachella Valley Multispecies Habitat Conservation Plan and the Conceptual Area Protection Plan.
	\\
	\hline
\end{longtabu}
\scalegraphics{lacertila/phrynosomatidae/uma/1}{0.15}
\scalegraphics{lacertila/phrynosomatidae/uma/2}{0.25}
\end{center}