\subsubsection{Urosaurus and Uta --- Tree and Side-Blotched Lizards}
\begin{center}
\begin{longtabu} to \textwidth {| | p{3.5cm} | X | |}

	\hline
	Taxonomy/Ancestry &
	
	 \\
	\hline
	Size & 
	\begin{itemize}[noitemsep]
		\item uta: males up to 60 mm (2.4 in), females a little smaller
		\item urosaurus: 4.7-6.6 cm (1 7/8-2 3/5 in)
	\end{itemize}
	\\
	\hline
	Color &
	\begin{itemize}[noitemsep]
		\item in uta, the degree of pigmentation varies w/ sex and population
			\begin{itemize}[noitemsep]
				\item some males have blue flecks spread over backs and tails, sides may be yellow or orange, others may be unpatterned
				\item females may have stripes along dorsal/sides or be plain
				\item prominent blotch on sides just behind front limbs
				\item blue-black mark present on sides of chest behind front limbs
			\end{itemize}
		\item urosaurus is greylish, light brown, or beige, and is able to quickly change from light to dark
	\end{itemize}
	 \\
	\hline
	Anatomy &
	small scales, larger keeled scales down the back. gular fold* across throat. long thin tales. live for about a year.
	 \\
	\hline
	Dimorphism & 
	see color.
	\\
	\hline
	Behavior & 
	\begin{itemize}[noitemsep]
		\item diurnal, tolerant of heat
		\item active from:
			\begin{itemize}[noitemsep]
				\item uta: active whenever temp is warm (all yr in southern deserts, semi-arid regions), inactive when cold
				\item urosaurus: active from Mar-fall
			\end{itemize}
		\item basking
			\begin{itemize}[noitemsep]
				\item uta: often found basking on rocks
				\item urosaurus: often found basking on lower branches
			\end{itemize}
		\item defense
			\begin{itemize}[noitemsep]
				\item uta: relies on coloring, so isn’t very wary; also autotomy
				\item urosaurus: relies on camouflage to disguise, runs away or to other side of branch when spotted
			\end{itemize}
		\item uta is known for its \textbf{morphs}
			\begin{itemize}[noitemsep]
				\item \textbf{orange-throated male} --- ultradominant, establish territory w/ multiple females, largest/most aggressive
				\item \textbf{blue-throated male} --- dominant, intermediate size, guard smaller territory w/ 1 female; better at catching yellow-throated but vulnerable to having female stolen by orange-throated
				\item \textbf{yellow-throated male} --- sneakers, mimic females to steal mates from orange-throated, smaller size; may under specific circumstances transform into blue-throated
				\item because 1 male morph does particularly well against another, but poorly against the 3rd (rock-paper-scissors), a cycle is created where the least common morph of 1 breeding season has the most offspring for the next
				\item \textbf{orange-throated female} --- r-strategists producing large clutches w/ many small eggs; more successful at lower population densities w/ less competition, less predators
				\item \textbf{yellow-throated female} --- – k-strategists producing small clutches w/ larger eggs; more successful at higher population densities w/ high predation
			\end{itemize}
	\end{itemize}
	\\
	\hline
	Habitat & 
	prefer open rocky areas w/ scattered vegetation. utilizes a wide variety of habitats, including hardpan, sandy, rocky, and loamy areas grown with chaparral, scattered trees, grass, shrubs, and cactus. urosaurus: favor creosote bushes w/ large exposed roots and spends night in burrows under shrub or in sand or at tips of branches, occasionally foraging on the ground.
	\\
	\hline
	Distribution & 
	ranges through most of California south of the Bay Area, all of Nevada, eastern Oregon, southwestern Idaho, central Washington, most of Utah, the western edge of Colorado, much of New Mexico the west part of Texas, north-central Mexico, along the west coast of Sonora, all of Baja California and many of its islands
	\\
	\hline
	Feeding Ecology & 
	insectivorous – beetles, ants, spiders, scorpions, ticks. may eat plants for water or by accident.
	\\
	\hline
	Reproductive Biology & 
	mate in spring, producing 1-7 clutches of 1-8 eggs laid March-August. females can store sperm to fertilize eggs later. juveniles hatch June-September, breeding the following spring.
	\\
	\hline
	Ecological Role &
	
	\\
	\hline
	Conservation Status & 
	
	\\
	\hline
\end{longtabu}
\textbf{Urosaurus:}

\scalegraphics{lacertila/phrynosomatidae/urosaurus-uta/1}{0.45}
\scalegraphics{lacertila/phrynosomatidae/urosaurus-uta/2}{0.25}

\textbf{Uta:}

\scalegraphics{lacertila/phrynosomatidae/urosaurus-uta/3}{0.35}
\scalegraphics{lacertila/phrynosomatidae/urosaurus-uta/4}{0.25}
\end{center}