\subsubsection{Cophosaurus and Holbrookia --- Earless Lizards}
\begin{center}
\begin{longtabu} to \textwidth {| | p{3.5cm} | X | |}

	\hline
	Taxonomy/Ancestry &
	cophosaurus = greater earless lizards. 1 species: \emph{C. texanus}
	
	holbrookia = lesser eared lizards, named for zoologist John Edwards Holbrook. 3 recognized species.
	 \\
	\hline
	Size & 
	greater: $2 \frac{3}{4}$ to $7 \frac{1}{4}$ in
	
	lesser: up to 70 mm (2.75 in)
	\\
	\hline
	Color &
	skin coloration typically matches habitat soil, sprinkled w/ light brown dots. the greater's tail is mostly black underneath, and the lesser's tail is plain white underneath. in the lesser, 2 black bars mark the lower side of the body around the forelimb.
	 \\
	\hline
	Anatomy &
	
	 \\
	\hline
	Dimorphism & 
	in the greater, the male has 2 distinct black lines anterior to the hind legs that wrap onto the ventral side and stop abruptly. females/juveniles have a distinct dark stripe on the back side of each thigh. pregnant females have lighter coloration on the flanks. 
	
	in the lesser, 2 black bars mark the lower side of the body around the forelimb. females often have a peach/pink tint w/ an orange throat patch.
	\\
	\hline
	Behavior & 
	\begin{itemize}[noitemsep]
		\item both are diurnal
		\item the greater is extremely active at all times of day, even the hottest hours
			\begin{itemize}[noitemsep]
				\item only hides during cloudy days
				\item fast; raises tail as it runs and waves tail when it slows or halts
				\item rarely stops on flat, open ground; prefers rocks/boulders
				\item coloration blends into rocks and soil
				\item not very wary; can stop for long time before running from approacher
			\end{itemize}
	\item the lesser is most active in the mid-morning and late afternoon during the hottest summer months
		\begin{itemize}[noitemsep]
				\item hibernates during cold months of winter, late fall
		\end{itemize}
	\end{itemize}
	\\
	\hline
	Habitat & 
	\begin{itemize}[noitemsep]
		\item the greater is terrestrial, inhabiting deserts/dunes and rocky areas such as desert flats, streambeds, limestone cliffs
		\item the lesser has communities ranging from semidesert grassland, interior chaparral, into woodlands
			\begin{itemize}[noitemsep]
				\item enters Arizona upland Sonoran desertscrub in some localities
				\item usually encountered on level terrain in open, sunlit areas w/ sparse vegetation, sandy/gravelly soil
			\end{itemize}
	\end{itemize}
	\\
	\hline
	Distribution & 
	\begin{enumerate}
		\item greater: found west of Fort Worth/Austin to eastern Trans-Pecos area in Texas, into Arizona and N.  Mexico. absent from eastern Texas, Lower Rio Grande Valley, Panhandle
		\item lesser: Distributed across northeastern plateau region, portion of Arizona strip, sub-Mogollon rim central Arizona, and northernmost sky islands in SE portion. found at elevation  ranging from 2,200' to 7,000'.
	\end{enumerate}
	\\
	\hline
	Feeding Ecology & 
	insectivores. consume grasshoppers, beetles, bees, wasps, ants, butterflies, moths, spiders, small lizards.
	\\
	\hline
	Reproductive Biology & 
	\begin{enumerate}
		\item greater: lay eggs Mar-Aug, take ~50 days to hatch; 3 clutches/season. rarely reaches 2 years of age.
		\item lesser: spring mating, 1-2 clutches laid in spring/summer. 1-10 eggs/clutch.
	\end{enumerate}
	\\
	\hline
	Ecological Role &
	
	\\
	\hline
	Conservation Status & 
	considered LC or NT.
	\\
	\hline
\end{longtabu}
\scalegraphics{lacertila/phrynosomatidae/"cophosaurus-holbrookia"/1}{0.25}
\scalegraphics{lacertila/phrynosomatidae/"cophosaurus-holbrookia"/2}{0.45}
\end{center}