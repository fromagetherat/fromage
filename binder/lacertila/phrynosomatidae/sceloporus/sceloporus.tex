\subsubsection{Sceloporus --- Spiny Lizards}
\begin{center}
\begin{longtabu} to \textwidth {| | p{3.5cm} | X | |}

	\hline
	Taxonomy/Ancestry &
	\centeredgraphics{lacertila/phrynosomatidae/sceloporus/tax}{0.5}
	 \\
	\hline
	Size & 
	up to 5.6 in
	\\
	\hline
	Color &
	base coloration = grey, tan, brown.
	\begin{enumerate}
		\item adult male --- blue/violet patches on belly + throat, green/blue color on tail + sides
		\item female/juvenile --- large combined dark spots on black/belly areas
		\item brownish/yellow triangular spots on shoulders
		\item winter = darker colors to absorb heat
		\item summer = lighter to reflect radiation
	\end{enumerate}
	 \\
	\hline
	Anatomy &
	large, pointed, keeled*, overlapping scales w/ sharp spines
	 \\
	\hline
	Dimorphism & 
	see color.
	\\
	\hline
	Behavior & 
	\begin{itemize}[noitemsep]
		\item adjust internal temp by changing color
		\item camouflage
		\item basks on rocks/hard surfaces
		\item shelters underground in burrows or under cover during hottest part of day in summertime
		\item hibernates in late fall and during cold months of winter, re-emerges in spring
		\item good climber
		\item males territorial, stand tall, expose blue throat, and do push-up display
		\item autotomy
	\end{itemize}
	\\
	\hline
	Habitat & 
	\begin{itemize}[noitemsep]
		\item biotic communities like Sonoran desertscrub, Great Basin desertscrub, semidesert grassland, interior chaparral, and woodlands
		\item usually encountered on lower slopes
		\item in tree branches or vicinity of ground cover (eg wood piles, rock piles, packrat nests)
	\end{itemize}
	\\
	\hline
	Distribution & 
	ranges across deserts of southwestern Arizona and northeastern plateaus. elevations from near sea level around Colorado River to 5000’.
	\\
	\hline
	Feeding Ecology & 
	\begin{itemize}[noitemsep]
		\item feeds on insects
		\item ants, beetles, caterpillars, spiders, centipedes, small lizards
		\item occasionally small lizards, nesting birds, leaves, flowers, berries
	\end{itemize}
	\\
	\hline
	Reproductive Biology & 
	\begin{itemize}[noitemsep]
		\item sexually mature at 2-3 years
		\item breed in spring, early summer --- generally May/June, sometimes until August
		\item clutch of 3-19 eggs laid May-August
		\item hatch August-September, sometimes October
		\item females lay more than 1 clutch
	\end{itemize}
	\\
	\hline
	Ecological Role &
	
	\\
	\hline
	Conservation Status & 
	no issues.
	\\
	\hline
\end{longtabu}
\scalegraphics{lacertila/phrynosomatidae/sceloporus/1}{0.15}
\scalegraphics{lacertila/phrynosomatidae/sceloporus/2}{0.15}
\end{center}