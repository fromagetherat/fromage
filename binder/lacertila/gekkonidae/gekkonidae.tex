\subsection{Gekkonidae --- Gecko Lizards}
\begin{center}
\begin{longtabu} to \textwidth {| | p{3.5cm} | X | |}

	\hline
	Taxonomy/Ancestry &
	part of the infraorder Gekkota. gekkonidae is the largest family of geckos, with over 950 described species in 51 
	
	\centeredgraphics{lacertila/gekkonidae/tax}{0.5}
	 \\
	\hline
	Size & 
	
	\\
	\hline
	Color &
	
	 \\
	\hline
	Anatomy &
	\begin{itemize}[noitemsep]
		\item ectothermic*
		\item shed skin regularly; detach loose skin from body and eat it
		\item 60\% of gecko species have adhesive toe pads that have been gained and lost repeatedly thru evolution
			\begin{itemize}[noitemsep]
				\item spatula-shaped setae (bristles found on the toe pads) arranged in lamellae (thin plate-shaped structures) enable attractive Van der Waals' forces b/w beta-keratin lamellae/setae/spatulae structures and surface
				\item self-cleaning; can remove dirt just by stepping
				\item does not adhere to teflon (polytetrafluroethene) b/c it has low surface energy
				\item humidity fortifies gecko tension even on hydrophobic surfaces, but tension is reduced if completely immersed in water
				\item molecular water layers carry large dipole moment; when present on setae and surface, surface energy of both is increased, therefore energy gain of contacting surfaces is increased, resulting in higher gecko adhesion force
				\item elastic properties of beta-keratin change w/ water uptake
				\item phospholipids lubricate setae and allow gecko to detach foot for next step
				\item every square mm of footpad = 14,000 setae
			\end{itemize}
		\item not double-jointed, but display digital hyperextension --- toes can hyperextend in opposite directions from human fingers and toes
		\item skin is a papillose surface made from hair-like protuberances developed across entire body (no scales), conferring superhydrophobicity; antimicrobial action
		\item polyphodonts* --- replace each of 100 teeth every 3-4 months
			\begin{itemize}[noitemsep]
				\item next to each full grown tooth there is a small replacement tooth developing
				\item formation of teeth = pleurodont: fused/ankylosed by sides to inner surface of jaw bones
			\end{itemize}
		\item instead of eyelids, geckos have a transparent membrane which they lick to clean
		\item nocturnal species = excellent night vision 350x more sensitive than humans; 3 different photopigments sensitive to UV, blue, and green
	\end{itemize}
	 \\
	\hline
	Dimorphism & 
	
	\\
	\hline
	Behavior & 
	\begin{itemize}[noitemsep]
		\item unique among lizards for their vocalizations: they use chirping sounds in social interaction
			\begin{itemize}[noitemsep]
				\item use to defend important resources (e.g. feeding areas)
			\end{itemize}
		\item mostly nocturnal --- emerge from hiding places in early evening to forage/seek mates, but their body temperature drops as the night progresses, limiting activity
		\item in diurnal species, there are 1 or 2 peaks of activity during day, often in late morning and mid-to-late afternoon
		\item tropical species are active year-round
		\item northern and southern species remain inactive within burrows during cold periods, though they may emerge during warmer nights
		\item solitary, but some species can reach high densities and share retreat sites. such species demonstrate reduced aggression towards each other but no complex social structure
		\item deter predators using vocalizations, bites, defecation
			\begin{itemize}[noitemsep]
				\item cryptic coloration or concealing skin folds/flaps to avoid detection
				\item some outrun predators
				\item some species can shed skin if grabbed
				\item demonstrate autotomy*, but can only use as last resort b/c tail stores nutrients
					\begin{itemize}[noitemsep]
						\item decreased activity following autotomy to recover
						\item often return to area where they lost it; if still there, they eat it
						\item some species attack rivals and eat their tail
					\end{itemize}
			\end{itemize}
	\end{itemize}
	\\
	\hline
	Habitat & 
	require egg-laying sites, adequate supplies of arthropod prey, and retreats protecting against temp extremes and predators. often substrate-limited and need certain kinds of rocks. arid zones = narrow rock crevices or burrow. humid tropical forest habitats also common --- trunks, branches, tree canopies, rotting logs, rocks. savannas/grasslands = less numerous, patchy distributions --- trees, rocks, nests. several species live inside human habitations in warm parts of the world --- often welcomed b/c they feed on insects and artificial lighting attracts prey.
	\\
	\hline
	Distribution & 
	\begin{itemize}[noitemsep]
		\item Migrated over the world from Pacific Rim 1000s of yrs ago
		\item Spread across islands and continents
		\item Chiefly tropical and subtropical but range as far north as southwestern US, southern Europe, southern Siberia
		\item Reach New Zealand and approach southern tip of S. America to south
		\item Most species restricted to small geographic ranges
	\end{itemize}
	\\
	\hline
	Feeding Ecology & 
	\begin{itemize}[noitemsep]
		\item often eat eat insects (eg moths, mosquitoes, crickets, grasshoppers, mealworms)
		\item larger species take small vertebrate prey (eg small snakes, lizards, mammals, birds)
		\item some island lizards supplement diet w/ fruits, nectar, or pollen; these lizards play roles as pollinators/seed dispensers
		\item hunt using combo of visual/chemical clues
		\item other species = ambush predators
	\end{itemize}
	\\
	\hline
	Reproductive Biology & 
	\begin{itemize}[noitemsep]
		\item some species parthenogenetic --- female can reproduce w/o copulating w/ male
			\begin{itemize}[noitemsep]
				\item improves ability to spread to new islands
				\item islands populated by single female gecko = lack of genetic diversity
				\item sometimes there are no males at all due to hybridization of 2 bisexual parent species
			\end{itemize}
		\item less vocal geckos can identify members of opposite sex thru chemical cues or vision
		\item males rub/lick females before mating; restrain during copulation by biting nape of neck or back
		\item most lay eggs
			\begin{itemize}[noitemsep]
				\item lay eggs in protected sites providing high-humidity microclimate such as under bark, in shallow nests, or burrows/rock crevices
				\item fixed clutch sizes --- mostly 2, sometimes 1
				\item tropical species may produce several clutches, species in colder areas usually have only 1
				\item typically abandon eggs
			\end{itemize}
		\item development = 1-6 months depending on temp
		\item some have TSD: high temp = male, low temp = female
		\item hatchlings slit shells w/ paired egg teeth shed shortly after eclosion
		\item geckos of New Zealand and 1 species in New Caledonia = viviparous, possess simple placenta. always produce twins which gestate for 4-14 months
	\end{itemize}
	\\
	\hline
	Ecological Role &
	
	\\
	\hline
	Conservation Status & 
	population estimates exist for only a few ? conservation status of most species unknown. island-dwelling geckos w/ limited distributions threatened by habitat destruction (ie deforestation), introduction of rats, cats, predatory mammals. only extinct known geckos = largest geckos, giant gecko of Round Island and Delcourt?s giant gecko from New Zealand. large geckos once hunted for food. most modern consumption = medicinal. sold dried or pickled to increase vitality and cure ailments in China and SE Asia.
	\\
	\hline
\end{longtabu}
\scalegraphics{lacertila/gekkonidae/1}{0.5}
\scalegraphics{lacertila/gekkonidae/3}{0.5}
\scalegraphics{lacertila/gekkonidae/2}{0.15}
\end{center}