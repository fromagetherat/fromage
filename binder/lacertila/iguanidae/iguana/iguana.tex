\subsubsection{Iguana --- Green Iguana}
\begin{center}
\begin{longtabu} to \textwidth {| | p{3.5cm} | X | |}

	\hline
	Taxonomy/Ancestry &
	subfamily Iguaninae. 2 species, widespread green iguana and endangered Lesser Antillean iguana.
	\centeredgraphics{lacertila/iguanidae/iguana/tax}{0.5}
	 \\
	\hline
	Size & 
	1.5-1.8 m (5-6 ft) in length including tail. avg. male = 4 kg (8.8 lb); avg. female = 1.2-3 kg (2.6-6.6 lb). 
	\\
	\hline
	Color &
	dewlap typically orange. yellow eyes. despite the name, green iguanas come in a variety of colors, including green to lavender,  blue, black, pink.
	 \\
	\hline
	Anatomy &
	\begin{itemize}[noitemsep]
		\item row of spines running from back to tail
		\item \textbf{parietal eye}* --- tiny 3rd eye on head resembling a pale scale. functions as a light-sensing organ to help detect predators stalking from above.
		\item \textbf{tuberculate scales}* --- small scales resembling spokes behind necks
		\item \textbf{sub-tympahnic scale}* --- large round scale on cheeks located below tymphanum (eardrum) behind each eye
		\item 3-chambered heart w/ 2 atria, 1 ventricle, 2 aortae w/ systemic circulation like most reptiles 
		\item skull and body show adaptations to herbivorous lifestyle (strong bite, efficient processing)
			\begin{itemize}[noitemsep]
				\item taller/wider skulls
				\item shorter snouts
				\item larger bodies
				\item \textbf{acrodontal teeth}* --- sit on top of surface of jawbone, project upwards. small and serrated to hold food
			\end{itemize}
		\item adults found on St. Lucia have many differences compared to other green iguanas
			\begin{itemize}[noitemsep]
				\item light green w/ predominant black stripes
				\item black dewlap
				\item eyes white or cream
				\item smaller scales to back of head near jawbone
				\item typically lay 25 eggs instead of usual 50
			\end{itemize}
		\item lateral nasal gland to supplement renal salt excretion by expelling excess potassium and sodium chloride. not capable of creating liquid urine more concentrated than bodily fluids, so they excrete nitrogenous wastes as urate salts thru salt gland.
	\end{itemize}
	 \\
	\hline
	Dimorphism & 
	males have 2 hemepenes. male has more developed femoral pores; longer and thicker spines
	\\
	\hline
	Behavior & 
	\begin{itemize}[noitemsep]
		\item navigate thru crowded forests using visual acuity to locate food
		\item employ visual signals to communicate w/ others of same species
		\item whip-like tails can be used to attack or for autotomy
		\item dewlap helps regulate body temp, used in courtship and territorial displays
		\item attempt to flee when frightened
			\begin{itemize}[noitemsep]
				\item may dive into nearby body of water and swim away
				\item if cornered, will extend dewlap, stiffen/puff up body, hiss, and bob head @ predator
				\item can lash w/ tail, teeth, claws; wounded more inclined to fight
			\end{itemize}
		\item use head bobs and dewlaps in social situations --- greeting/courting; frequency and \# of head bobs = special meaning
		\item males often use bodies to shield females from predators --- only species of reptile that does this
	\end{itemize}
	\\
	\hline
	Habitat & 
	arboreal, but often found near water. climb up trees but stay near ground during colder weather.
	\\
	\hline
	Distribution & 
	native to tropical areas of Mexical, Central America, S. America, and Caribbean.
	\\
	\hline
	Feeding Ecology & 
	\begin{itemize}[noitemsep]
		\item primarily herbivores
		\item require precise ratio of minerals --- 2:1 calcium to phosphorus 
		\item forage exclusively on vegetation and foliage --- turnip greens, mustard greens, dandelion greens, wild flowers, fruit, growing shoots of $>$100 species of plant, wild plums, collards, butternut squash, acorn squash, mango, parsnip
		\item juveniles often eat feces from adults to acquire microflora to process low-quality herbivorous foraging diet
		\item some debate as to whether captive animals should be fed animal protein --- can result in renal failure and other health problems
		\item hunted by predatory birds
	\end{itemize}
	\\
	\hline
	Reproductive Biology & 
	\begin{itemize}[noitemsep]
		\item femoral pores secrete scent
		\item males display dominant behaviors
		\item oviparous
		\item clutches of 20-71 eggs once per year during synchronized nesting 
		\item no parental protection after laying
		\item in Panama, green iguana has been observed sharing nesting sites w/ American crocodiles and in Honduras w/ spectacled caimans
		\item hatchlings emerge after 10-15 wks incubation
		\item juveniles stay in familial groups for 1st year 
	\end{itemize}
	\\
	\hline
	Ecological Role &
	
	\\
	\hline
	Conservation Status & 
	lesser antillean endangered. enforcement of hunting regulations difficult, suffer from habitat loss to agriculture, predation by introduced animals, and competition from the invasive green iguana. iguanas sometimes used as food source.
	\\
	\hline
\end{longtabu}
\scalegraphics{lacertila/iguanidae/iguana/1}{0.75}
\scalegraphics{lacertila/iguanidae/iguana/2}{0.15}
\end{center}