\subsubsection{Dipsosaurus}
\begin{center}
\begin{longtabu} to \textwidth {| | p{3.5cm} | X | |}

	\hline
	Taxonomy/Ancestry &
	monotypic genus --- \emph{D. dorsalis}. originates from ``dipsa" --- thirsty --- and ``saurus" --- lizard. ``dorsalis" = Latin ``dorsum" = spike.
	
	\centeredgraphics{lacertila/iguanidae/dipsosaurus/tax}{0.5}
	 \\
	\hline
	Size & 
	blunt, medium-sized --- 61 in (24 cm) long
	\\
	\hline
	Color &
	Pale grey-tan to cream w/ light brown reticulated (net) pattern on backs and sides. reticulated pattern becomes brown spots near back legs, then tail stripes. sides become pinkish during breeding season.
	 \\
	\hline
	Anatomy &
	row of keeled dorsal scales going down back (become slightly larger over back).
	 \\
	\hline
	Dimorphism & 
	
	\\
	\hline
	Behavior & 
	scamper into shrub and go down burrow when threatened.
	\\
	\hline
	Habitat & 
	\begin{itemize}[noitemsep]
		\item habitat confined within range of the creosote bush
		\item dry, sandy desert scrubland below 1,000 m (3,000 ft)
		\item can also be in rocky streambeds up to 1,000 m
		\item southern portion = areas of arid subtropical scrub and tropical deciduous forest
		\item can withstand high temp, often out after other lizards shelter
		\item use burrows
			\begin{itemize}[noitemsep]
				\item dug in sand under bushes
				\item often borrow burrows of kit foxes and desert tortoises
			\end{itemize}
	\end{itemize}
	\\
	\hline
	Distribution & 
	Sonoran and Mojave deserts of southwestern US and northwestern Mexico. occur on several Gulf of California islands.
	\\
	\hline
	Feeding Ecology & 
	herbivorous. buds, fruits, leaves of annual and perennial plants, especially yellow creosote bush flowers.
	\\
	\hline
	Reproductive Biology & 
	mate in early spring, producing 1 clutch of eggs per year of 3-8 eggs. hatchlings emerge around September.
	\\
	\hline
	Ecological Role &
	consumed by birds of prey, foxes, rats, long-tailed weasels, some snakes, humans.
	\\
	\hline
	Conservation Status & 
	LC. sometimes used as a meat source by humans.
	\\
	\hline
\end{longtabu}
\scalegraphics{lacertila/iguanidae/dipsosaurus/1}{0.25}
\scalegraphics{lacertila/iguanidae/dipsosaurus/2}{1.5}
\end{center}