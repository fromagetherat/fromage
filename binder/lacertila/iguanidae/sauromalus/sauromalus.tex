\subsubsection{Sauromalus --- Chuckwalla} 
\begin{center}
\begin{longtabu} to \textwidth {| | p{3.5cm} | X | |}

	\hline
	Taxonomy/Ancestry &
	6 species. comes from Greek ``sauros" --- lizard --- and ``omalus" --- flat. ``chuckwalla" comes from Shoshone ``tcaxxwal" or Cahuilla ``caxwal" transcribed by Spaniards as ``chacahuala"
	
	\centeredgraphics{lacertila/iguanidae/sauromalus/tax}{0.5}
	 \\
	\hline
	Size & 
	common chuckwalla measures $15 \frac{3}{4}$ in. long
	\\
	\hline
	Color &
	see sexual dimorphism.
	 \\
	\hline
	Anatomy &
	\begin{itemize}[noitemsep]
		\item stocky, wide-bodied w/ flattened midsections, prominent bellies
		\item may live for 25 yrs or more
		\item thick tails tapering to blunt tip
		\item loose folds of skin along neck and body
		\item small, coarsely granular scales
	\end{itemize}
	 \\
	\hline
	Dimorphism & 
	males --- reddish-pink to orange, yellow, or light grey bodies + black heads, shoulders, and limbs; larger and possess well-developed femoral pores on inner sides of thighs.
	
	females/juveniles --- bodies w/ scattered spots or contrasting bands of light/dark in shades of grey or yellow.
	\\
	\hline
	Behavior & 
	\begin{itemize}[noitemsep]
		\item run from potential threats and wedge into tight rock crevice and inflate self
		\item males seasonally + conditionally territorial
			\begin{itemize}[noitemsep]
				\item abundant resources = hierarchy based on size
				\item combo of color + physical displays (e.g. push-ups, head-bobbing, gaping of mouth)  communicate/defend territory
			\end{itemize}
		\item diurnal
		\item ectothermic* --- spend most of mornings/winter days basking
		\item hibernate during cooler months, emerge in February
	\end{itemize}
	\\
	\hline
	Habitat & 
	prefer lava flows + rocky areas vegetated by creosote bush and other drought-tolerant scrub. may be found at elevations up to 4,500 ft (1,370 m)
	\\
	\hline
	Distribution & 
	wide distribution in biomes of Sonoran + Mojave deserts. common chuckwalla (S. ater) = greatest range – from SoCal east to southern Nevada and Utah and Western Arizona, and south to Baja California + northwestern Mexico. peninsular chuckwalla (S. australis) found on eastern portion of southern half of Baja California Peninsula. other species island-dwelling found off coast of Baja California or in Gulf of California, believed to have been translocated to some islands by Comca’ac (Seri) ppl as food source.
	\\
	\hline
	Feeding Ecology & 
	primarily herbivorous, feed on leaves, fruit, flowers of annuals + perennial plants. they are said to prefer yellow flowers. insects = supplementary prey.
	\\
	\hline
	Reproductive Biology & 
	mating takes place from April-July. 5-16 eggs are laid b/w June and August, which hatch in late September.
	\\
	\hline
	Ecological Role &
	fed on by coyotes/other mammals, larger avian predators, snakes.
	\\
	\hline
	Conservation Status & 
	2 LC; 1 NT; 2 EN; 1 NE. Angel Island species eaten by Comca’ac (Seri) ppl.
	\\
	\hline
\end{longtabu}
\begin{figure}[t]
\centering	\scalegraphics{lacertila/iguanidae/sauromalus/1}{0.8}
	\caption{Male chuckwalla}
	\scalegraphics{lacertila/iguanidae/sauromalus/2}{2.5}
	\caption{Female chuckwalla}
\end{figure}
\end{center}