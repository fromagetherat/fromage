\subsubsection{Anolis --- Anoles}
\begin{center}
\begin{longtabu} to \textwidth {| | p{3.5cm} | X | |}

	\hline
	Taxonomy/Ancestry &
	generally considered to be monotypic containing only Anolis, but recent genetic research identified several clades within Anolis which may sometimes be elevated to generic status: Dactyloa, Deiroptyx, Ctenonotus, Xiphosurus, Norops, Chamaelinorops, Anolis, Audantia. genus Polychrus was previously placed in the family under family name Polychrotidae, but recent genetic studies confirm it is not closely related and is now invalidated and classified as Polychrus in family Iguanidae
	
	391 species in anolis. displays considerable paraphyly but phylogenetic analysis suggests some subgroups/clades. several species of Anolis occasionally prescribed to proposed genus Norops but validity of Norops is sketchy.
	
	known for being remarkably adaptable --- rapidly adapt behavior/morphology over ecological timescales. Presence of ground predator = selective gradient in favor of longer hindlimbs within a generation, followed by shorter hindlimbs as they tended to perch higher up. Cuban anole living in Florida rapidly adapted, as did native anoles. reen anole moved to higher perches and adapted large toepads better suited for those --- observed in just 20 generation.
	
	\centeredgraphics{lacertila/polychridae/anolis/tax}{0.5}
	 \\
	\hline
	Size & 
	8-18 cm (3-7 in); some larger species can surpass 12 in or even reach 20 in
	\\
	\hline
	Color &
	large majority sport green coloration (only species native to US = green anole). many can change color to a limited extent (only changing to 1 color), but the extent of the ability varies widely between species. they are often referred to as ``American chameleons."
	 \\
	\hline
	Anatomy &
	\begin{itemize}[noitemsep]
		\item live b/w 4-8 yrs but may live beyond w/ proper care
		\item males have \textbf{dewlaps} made of erectile cartilage extending from neck and throat areas (see behavior)
		\item pads w/ several flaps of skin horizontally covered in microscopic hair-like protrusions (setae) which allow them to cling to surfaces like a gecko
	\end{itemize}
	 \\
	\hline
	Dimorphism & 
	green anole = female has pale dorsal stripe extending from neck to tail, smaller body, smaller head w/ shorter snout.
	
	brown anoles = share above characteristics w/ wider dorsal stripe, often diamond-shaped or w/ squiggly edges.
	
	stripes may be present in males, but always smaller and fainter. some females have pale, v small dewlaps.
	\\
	\hline
	Behavior & 
	\begin{itemize}[noitemsep]
		\item change color based on stress level, sun/light exposure, surroundings
		\item use dewlaps as signal for attracting mates, winning contests, communicating condition
		\item diurnal
		\item utilize autotomy to escape predators at times
		\item usually have small territories w/ basking area, shady area, high lookout, and place to hide
			\begin{itemize}[noitemsep]
				\item do not tolerate other anoles within territory
				\item raises spine, fans dewlap, does ``push-ups" accompanied by hisses
				\item males will fight by biting each others' tails
			\end{itemize}
	\end{itemize}
	\\
	\hline
	Habitat & 
	semiarboreal, they usually inhabit regions 3-6 m (9.8-19.7 ft) high such as shrubs, walls, fences, bushes, short trees.
	\\
	\hline
	Distribution & 
	found throughout southeastern US, @ least as far west as San Antonio, Caribbean, Mexico, and other warm regions of western world. knight, green (only native), bark, Jamaican giant, and Cuban brown anoles can all be found in US, primarily Florida. most prevalent = Cuban brown, pushed native green/Carolina anole pop. further north. when green and brown inhabit same area, brown = primarily terrestrial/lower branches, green anoles = higher.
	\\
	\hline
	Feeding Ecology & 
	live insects and other invertebrates, such as crickets, spiders, moths. opportunistic feeders --- eat any attractive meal that is small enough. 
	\\
	\hline
	Reproductive Biology & 
	breeding takes place for several months beginning in the late spring. males employ head-bobbing and dewlap extension. they lay 1-2 small, soft-shelled eggs in leaf litter. multiple clutches can be laid at a time.
	\\
	\hline
	Ecological Role &
	predators include skinks, cats, snakes, birds, sometimes other larger lizards. less susceptible to predation if they have a dewlap where both scales and skin in between match expected pale grey or white color of ventral surface.
	
	known for demonstrating ecomorphs --- species w/ same structural habitat/niche, similar in morphology in behavior, not necessarily close phyletically. they show both adaptive radiation and convergent evolution, repeatedly evolving into similar forms on different islands.
	\begin{itemize}[noitemsep]
		\item Crown-giant - large body, large head, large sub-digital lamallae, inhabit uppermost canopy 
		\item Grass-bush - upper most reaches of trunks of tall trees and lower canopy, predominantly green, large sub-digital toe pads and short stout legs to aid in arboreal locomotion, have most drastic color-changing
		\item Trunk - trunks of tall trees, mid-sized, short limbs/tails, all, short, triangular heads
		\item Trunk-crown
		\item Trunk-ground - perch on lower trunk of trees or rocks immediately under tree trunk, stocky w/ relatively large heads and long legs for jumping (jump onto prey on ground and retreat back into tree)
		\item Twig
		\item Primarily related to substrate diameter
	\end{itemize}
	\\
	\hline
	Conservation Status & 
	green anole A. carolinensis became 1st reptile to have complete genome published. some species in Caribbean threatened due to small range. many non-native anoles introduced to new areas by humans; may outcompete indigenous species. function well as native pest control. may bite humans but rarely draw blood.
	\\
	\hline
\end{longtabu}
\scalegraphics{lacertila/polychridae/anolis/1}{0.35}
\scalegraphics{lacertila/polychridae/anolis/2}{0.5}
\end{center}