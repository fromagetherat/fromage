\subsection{Trionychidae --- Soft-Shelled Turtles}
\begin{center}
\begin{longtabu} to \textwidth {| | p{3.5cm} | X | |}

	\hline
	Taxonomy/Ancestry &
	sometimes called ``pancake turtles." N. American members of genus \emph{Trionyx} assigned incorrect resurrected name \emph{Apalone} until 1987. 3 subfamilies: Plastomeninae (extinct); Cyclanorbinae; Trionychinae.
	
	most closely related to Carettochelydidae (pig-nosed turtles). fossils suggest much broader distribution than currently known. oldest member dates from late Jurassic.
	
	\centeredgraphics{testudines/trionychidae/tax}{0.5}
	 \\
	\hline
	Size & 
	
	\\
	\hline
	Color &
	
	 \\
	\hline
	Anatomy &
	\begin{itemize}[noitemsep]
		\item shell lacks horny scutes* (spiny softshell does have some protrusions) --- carapace = leathery and pliable
		\item central part of carapace has layer of solid bone, outer edges don't
		\item soft shell helps them move easily in open water or lake bottoms; faster on land
		\item feet = webbed, 3-clawed
		\item carapace color of each species tends to match sand/mud color of their region = camouflage for feeding
		\item sexual dimorphism --- females much larger than males
		\item many characteristics of aquatic lifestyle
			\begin{itemize}[noitemsep]
				\item must be submerged to swallow food
				\item necks disproportionately long to breathe surface air from water
				\item ``breathe" underwater w/ rhythmic movements of mouth containing numerous processes copiously supplied w/ blood like gill filaments in fish
				\item Chinese softshell shown to excrete urea while ``breathing" underwater --- efficient solution in brackish environments
			\end{itemize}
	\end{itemize}
	 \\
	\hline
	Dimorphism & 
	Females can grow up to several feet in carapace diameter, while males stay much smaller.
	\\
	\hline
	Behavior & 
	spend much time lying in mud. basking is not common. use a ``lie and wait" feeding methodology. much faster than other turtles due to light shell.
	\\
	\hline
	Habitat & 
	soft bottom bodies of freshwater, although they can also adapt to highly brackish waters. favor slow moving streams, swift rivers, lakes, ponds.
	\\
	\hline
	Distribution & 
	eastern N. America, Africa, Asia, and Indo-Australian archipelago.
	\\
	\hline
	Feeding Ecology & 
	most species are carnivorous, but some can be omnivorous. they eat crustaceans, insects, mollusks, fish, amphibians.
	\\
	\hline
	Reproductive Biology & 
	courtship observed in a few species, involves head bobbing and male rubbing female's carapace. our knowledge of their reproduction is poor. clutch size = ~20 eggs; multiple clutches per year.
	\\
	\hline
	Ecological Role &
	
	\\
	\hline
	Conservation Status & 
	some species critically endangered in Asia due to harvesting for food. most commonly consumed = Chinese softshell \emph{Pelodiscus sinesis}. turtle farms exist.
	
	in the US ``harvesting" softshells was legal in Florida until recently. as of 2009, only 1 turtle per person per day can be collected.
	\\
	\hline
\end{longtabu}
\scalegraphics{testudines/trionychidae/1}{0.5}
\end{center}