\subsection{Testudinidae --- Tortoises}
\begin{center}
\begin{longtabu} to \textwidth {| | p{3.5cm} | X | |}

	\hline
	Taxonomy/Ancestry &
	\begin{itemize}[noitemsep]
		\item in America:
			\begin{itemize}[noitemsep]
				\item turtle is used as a general term for Testudines, including tortoises as a specific term for terrestrial turtles or specific members of Testudinidae
				\item terrapins are turtles that are small and live in fresh and brackish water
			\end{itemize}
		\item in Britain:
			\begin{itemize}[noitemsep]
				\item turtle not generic term for members of order Testudines, also applies ``tortoise" broadly to land-dwelling members
				\item ``terrapin" = larger group of semi-aquatic turtles
			\end{itemize}	
	\end{itemize}
	
	\centeredgraphics{testudines/testudinidae/tax}{0.5}
	 \\
	\hline
	Size & 
	
	\\
	\hline
	Color &
	
	 \\
	\hline
	Anatomy &
	\begin{itemize}[noitemsep]
		\item the number of concentric rings on the carapace can be used to tell age
			\begin{itemize}[noitemsep]
				\item growth depends highly on accessibility of food and water; well-fed tortoise w/o seasonal variation = no rings
				\item some tortoises grow >1 ring per season
				\item some have no rings visible due to wear
			\end{itemize}
		\item 1 of the longest animal lifespans; notable old tortoises: Tu?i Malila, Adwaita (oldest known, 255 if verified), Harriet, Timothy; typically 185 at max
		\item extremely small brains
			\begin{itemize}[noitemsep]
				\item Central/S. American tortoises have no hippocampus (emotion, learning, memory, spatial navigation)
				\item red-footed tortoises may rely on medial cortex
				\item Francisco Redi removed brain of land tortoise which then lived for 6 months; freshwater tortoises could but didn?t live as long. cut off head of tortoise which lived for 23 days.
			\end{itemize}
	\end{itemize}
	 \\
	\hline
	Dimorphism & 
	\begin{itemize}[noitemsep]
		\item Sometimes, males have longer, more protruding neck plate
		\item Sometimes, females have longer claws
		\item Female typically larger than male
		\item Male plastron curved concave to aid reproduction
		\item Females have smaller tails, dropped down; males have much longer tails which are normally pulled up and to the side of rear shell
	\end{itemize}
	\\
	\hline
	Behavior & 
	\begin{itemize}[noitemsep]
		\item move slowly on dry land at 0.17 mph (0.27 km/h). record speed = 5 mph (8.0 km/h)
		\item starts digging ground to create hibernaculum* at 1st signs of autumn
			\begin{itemize}[noitemsep]
				\item prefers swampy grounds where it can bury itself in mud
				\item loses appetite as temp drops
				\item may stop digging if temp increases but resumes immediately if it becomes cold
				\item stops breathing during hibernation
				\item wakes up in spring but only gradually regains appetite/energy as temp warms up
			\end{itemize}
		\item spend many hours sleeping in summer from late afternoon until the next morning
		\item love warm weather but avoid hot sun under green leaves/vegetation
		\item often engage in male-to-male combat
	\end{itemize}
	\\
	\hline
	Habitat & 
	terrestrial, from deserts and grasslands to shrublands and forest floors.
	\\
	\hline
	Distribution & 
	mainly tropical/subtropical in N./S. America, Europe, Asia, and Africa, as well as numerous oceanic islands.
	\\
	\hline
	Feeding Ecology & 
	mostly herbivorous, with some being omnivorous, they consume grasses, weeds, leafy greens, flowers, and some fruits. certain species consume worms, insects, and carrion. too much protein is detrimental in herbivores, causes shell deformities and medical problems. juveniles may have slightly different nutritional requirements (e.g. youth of herbivorous species will eat worms for protein). eat more in summer.
	\\
	\hline
	Reproductive Biology & 
	\begin{itemize}[noitemsep]
		\item courtship often involves male chasing female and ramming/biting her
		\item females dig nesting burrows for 1-30 eggs
		\item egg-laying occurs @ night 
		\item female covers clutch w/ sand, soil, organic material
		\item size of egg depends on mother, estimate by measuring width of cloaca b/w carapace and plastron
		\item after incubation, fully formed hatchling breaks shell w/ egg tooth
		\item hatch w/ embryonic egg sac for nutrition for 1st 3-7 days
	\end{itemize}
	\\
	\hline
	Ecological Role &
	
	\\
	\hline
	Conservation Status & 
	a notable VU species is the Galapagos giant tortoise, which is the largest living species whose lifespan can exceed 100 years. it was hunted almost to extinction for food but conservation/breeding efforts brought them back to VU.
	
	Kurma, a half-tortoise deity, is the 2nd avatar of Vishu in Hindu culture. they also serve as a symbol of longevity in Chinese culture.
	\\
	\hline
\end{longtabu}
\scalegraphics{testudines/testudinidae/1}{0.25}
\end{center}