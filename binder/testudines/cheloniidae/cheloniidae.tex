\subsection{Cheloniidae --- Sea Turtles}
\begin{center}
\begin{longtabu} to \textwidth {| | p{3.5cm} | X | |}

	\hline
	Taxonomy/Ancestry &
	7 species --- Green Sea Turtle, Loggerhead Sea Turtle, Olive Ridley Sea Turtle, Hawksbill Sea Turtle, Flatback Sea Turtle,  Green Sea Turtle, and the Kemp's Ridley Sea Turtle.
	
	\centeredgraphics{testudines/cheloniidae/tax}{0.5}
	 \\
	\hline
	Size & 
	71-213 cm in carapace length. around 350 lb.
	\\
	\hline
	Color &
	
	 \\
	\hline
	Anatomy &
	unlike tortoises and other turtles, they lack the ability to retract their heads into the shell. their plastron is considerably reduced from other turtle species, and connected to the top part of the shell by ligaments w/o a hinge separating the pectoral/abdominal plates of the plastron. they are the only turtles who front limbs are stronger than their back limbs. the carapace is oval/heart-shaped, and the limbs have been modified into flippers for swimming, so they cannot support the turtle's weight on land.
	 \\
	\hline
	Dimorphism & 
	
	\\
	\hline
	Behavior & 
	
	\\
	\hline
	Habitat & 
	living in tropical oceans, they spend most of their lives swimming out in the waters over the continental shelf, the neritic zone. they tend to frequent bays and estuaries.
	\\
	\hline
	Distribution & 
	far reaching into warmer temperatures and tropical/subtropical areas of Pacific and Atlantic ocean. also found in warmer seas such as Mediterranean seas.
	\\
	\hline
	Feeding Ecology & 
	omnivorous, but they mainly eat meat, such as sponges, jellyfish, mollusks, barnacles, sea urchins, even fish. they also eat algae and sea plants. 
	
	adults are predated upon by sharks, saltwater crocodiles, and coyotes or other canids may eat nesting females. eggs and hatchlings face predation from insects, crustaceans, mollusks, small mammals, birds, other reptiles, and various fish.
	\\
	\hline
	Reproductive Biology & 
	\begin{itemize}[noitemsep]
		\item courtship/mating takes place in shallow offshore waters
		\item male/female pairs float near the surface, w/ the male carapace protruding from the water
		\item females reproduce on multi-year cycles, but produce produce multiple clutches within single season (spring to late fall)
		\item 100 eggs/clutch
		\item incubation 50-60 days, warmer temp = faster development
		\item eggs tend to hatch at same time at night, possibly to aid in digging
		\item TSD: warm = female, cold = male
	\end{itemize}
	\\
	\hline
	Ecological Role &
	important role in marine ecosystems by maintaining health balance of sea grasses/reefs.
	\\
	\hline
	Conservation Status & 
	mostly endangered or threatened. 2 EN, 1 VU, 2 CE, 1 DD.
	
	they are mainly endangered due to their slow growth rate, such that many do not survive to adulthood. they are often caught by fisheries or fishermen or hunted for their eggs or shells. they may also develop tumors or deformities due to human pollution.
	\\
	\hline
\end{longtabu}
\scalegraphics{testudines/cheloniidae/1}{0.5}
\end{center}